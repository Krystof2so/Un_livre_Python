%%%%%%%%%%%%%%%%%%
%%% Chapitre 1 %%%
%%%%%%%%%%%%%%%%%%
\chapter{Apprendre à programmer}

\textit{Python} a l'avantage d'être un langage polyvalent, permettant le développement d'applications de bureau ou pour mobiles, comme le développement Web. Il bénéficie d'une documentation fournie, et de nombreux tutoriels et cours figurent sur le net, dans à peu près toutes les langues. La communauté \textit{Python} est importante et active, ainsi trouver de l'aide en ligne sera aisé. De plus c'est un des langages les plus faciles à apprendre. Au regard de tels avantages, nous pouvons dire que le langage \textit{Python} en fait un très bon langage pour qui veut débuter dans la programmation.
\medskip

Quand on débute dans l'apprentissage de la programmation il est souvent difficile de se centrer sur les concepts fondamentaux sans s'éparpiller. Un tel apprentissage nécessite donc de la rigueur en évitant de vouloir apprendre trop de choses en même temps. Il est nécessaire de rester centré sur un même concept, un même langage. Par la suite, quand les concepts de base seront maîtrisés avec un langage, il sera plus facile d'apprendre un autre langage, voire de s'orienter vers des concepts plus avancés.
\medskip

Apprendre à coder implique de s'investir, et donc de coder souvent. Un peu chaque jour et vous mesurerez mieux votre progression. Il faut donc, en même temps que l'on apprend de nouveaux concepts, pratiquer, soit à l'aide d'exercices, soit par le développement de mini projets. Cela va donc nécessiter du temps, jusqu'à plusieurs mois d'apprentissage.
\medskip

Une fois à l'aise avec l'ensemble des concepts de base, il faudra selon ses appétences se centrer sur un \textit{framework} afin de réaliser des projets concrets et avancés. Et là encore, du temps sera nécessaire. L'apprenti développeur pourra se centrer sur:
\begin{description}
    \item[Le \textit{framework Django}]: utilisé pour le développement \textit{WEB}.
    \item[Le \textit{framework Kivy}]: pour le développement d'applications mobiles et de bureau.
    \item[La \textit{data science}].
\end{description}
